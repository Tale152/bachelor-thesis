\section{Tecnologie coinvolte}
Essendo questo un applicativo web-based, sono state scelte delle tecnologie e API particolari in grado di svolgere questo compito.\newline
In particolare data la natura medio-ampia del progetto, che esula dai compiti più "tradizionali" di una pagina web, risultano necessarie soluzioni che permettono di ottenere una struttura solida e ben organizzata; quest'ultima si deve adattare ad una modellazione orientata agli oggetti, per meglio dividere le responsabilità delle componenti e, soprattutto, mantenere ed aggiornare lo stato di ogni componente ed i dati che da esso verranno utilizzati.\newline
Per soddisfare tali necessità si è deciso di optare per due framework ben consolidati: {\bf React} e {\bf Redux}.\newline
Attraverso queste due soluzioni risulta possibile trarre il massimo da JavaScript (in particolare \emph{Node Js}) con soluzioni semplici ed efficaci.
\subsection{React}
\noindent React, come citato sopra, permette di \emph{modellare il nostro codice seguendo i principi della programmazione ad oggetti}.
          In modo particolare risulta essenziale comprendere quattro concetti chiave:
\begin{itemize}
    \item  \textbf{JSX}\\
    JSX \cite{introduzione_jsx_react} è un'estensione della sintassi di JavaScript.\newline
    Permette di unire gli aspetti di html come linguaggio di template agli aspetti di JavaScript come linguaggio di scripting in una forma che ne aumenta semplicità e leggibilità.\newline
    Gli elementi JSX vengono utilizzati nelle definizioni delle funzioni di rendering (che verranno trattate a breve) semplificando la costruzione della user interface.\newline
    Attraverso JSX possiamo richiamare un componente React tramite con un meccanismo analogo ai tag in html.
    \item  \textbf{Componenti}\\
    Attraverso un componente \cite{componente_react} andiamo a definire quella che risulta essere a tutti gli effetti una classe.\newline
    Il nostro componente, definito da uno o più costruttori, contiene uno stato; questo stato verrà mantenuto, ed eventualmente aggiornato, durante tutto il ciclo di vita (lifecycle) del componente.\newline
    È possibile ricevere dati ed istruzioni da altri componenti attraverso le props.    
    \item \textbf{Stato}\\
    Lo stato \cite{state_e_lifecycle_react} un insieme di proprietà di un componente.\newline
    Queste proprietà possono variare a seguito dell'interazione con altri componenti o come azione del componente stesso nel caso esso esegua delle azioni a cadenza temporale.\newline
    In React inoltre lo stato risulta fondamentale ai fini di ottenere un rendering a schermo performante; ogni componente React infatti deve obbligatoriamente definire una funzione \emph{render()}.\newline
    Attraverso di essa verrà ritornato il contenuto da renderizzare a schermo.\newline 
    React cambierà il contenuto a schermo (consumando quindi risorse) solamente quando vi saranno delle modifiche nel contenuto ritornato dalla funzione render(); utilizzando dunque le proprietà che definiscono lo stato del componente all'interno della funzione render() del componente stesso riusciremo a ridurre al minimo indispensabile il numero di volte in cui l'applicazione verrà renderizzata, riflettendo i cambiamenti di stato del componente.    
    \vspace{10mm}
    \item \textbf{Lifecycle}\\
    All'interno del nostro componente possiamo (opzionalmente) definire due funzioni:
    \begin{itemize}
      \item \emph{componentDidMount()}\\
      Questa funzione sarà richiamata quando il componente verrà istanziato.\newline
      È qui che ad esempio verranno definiti dei timer attraverso i quali richiamare periodicamente delle funzioni.
      \item \emph{componentDidUnmount()}\\
      Analogamente questa funzione essa verrà richiamata quando il componente verrà distrutto.\newline
      Sarà dunque qui che ad esempio i timer periodici verranno fermati.
    \end{itemize}

\end{itemize}
\subsection{Redux}
Redux \cite{caratteristiche_redux} è un \emph{contenitore di stato} per applicazioni JavaScript.\newline
Gode di quattro caratteristiche fondamentali per progetti portata medio-grande:
\begin{itemize}
  \item \textbf{Deterministico}\\
  Un aspetto fondamentale nelle applicazioni web di ogni genere (e soprattutto in un contesto real-time) è il determinismo.\newline
  L'oneroso compito di far collaborare tutti i componenti al fine di ottenere un comportamento predicibile viene largamente semplificato dal framework Redux.
  \item \textbf{Centralizzato}\\
  Avere stato e logica centralizzati permette di ottenere rapidamente delle feature di fondamentale importanza, come funzioni di "annulla" e "ripeti" che ci permettono di muoverci agilmente tra lo storico degli stati.\newline
  Inoltre un approccio centralizzato garantisce un consistenza maggiore dei dati, rendendo possibile modificarli solamente attraverso specifiche funzioni (azioni) create durante la definizione della struttura dati.
  \item \textbf{Debug oriented}\\
  Attraverso semplici plugin browser come Redux DevTools risulta immediato il debug dell'applicazione.\newline
  Difatti attraverso questi tool otteniamo una visione completa dello stato dei dati e della sua evoluzione nel tempo, riuscendo ad identificare con precisione quando e soprattutto perchè lo stato abbia subito delle modifiche.
  \item \textbf{Flessibile}\\
  Redux funziona con ogni layer di User Interface e, essendo ormai uno strumento consolidato, dispone di un solido supporto e un vasto parco di plugin e pacchetti aggiuntivi per ogni esigenza.
\end{itemize}
Per comprendere come queste caratteristiche si concretizzino risulta fondamentale comprendere tre concetti chiave nella struttura di Redux:
\begin{itemize}
  \item \textbf{Store}\\
  Lo store \cite{store_redux} è un oggetto che contiene l'intera struttura ad albero dello stato.\newline
  Fornisce metodi per la lettura dello stato corrente.
  \item \textbf{Reducer}\\
  I reducer \cite{reducer_redux} definiscono la struttura dello store.\newline
  Vi possono essere più reducer all'interno di una stessa applicazione, al fine di meglio suddividere lo stato.
  \item \textbf{Azioni}\\
  Le azioni \cite{azioni_redux} permettono di modificare il contenuto dello store.\newline
  Essendo queste l'unico modo per alterare lo stato attuale, qualsiasi componente che voglia agire sullo stato deve passare per le azioni definite.
\end{itemize}
Ma \emph{perchè} utilizzare un contenitore di stato se ogni componente React possiede già uno stato?
React, pur essendo uno strumento molto potente, ci pone davanti a delle limitazioni:
\begin{itemize}
  \item
  Innanzitutto non è raro che più componenti debbano fare riferimento allo stesso dato; la presenza di uno stato comune evita di dover implementare meccanismi ad hoc di sincronia tra gli stati dei due componenti.\newline
  Lo stato di ogni componente, attraverso tool ready to use, sarà dunque sincronizzato con la struttura principale.
  \item
  In secondo luogo React incoraggia lo scambio di dati solo in una direzione: da un componente padre verso un componente figlio (utilizzato dunque dal padre).\newline
  La presenza delle azioni Redux risolvono questo problema in quanto ogni cambiamento apportato allo stato Redux si rifletterà su tutti i componenti React che utilizzano quel particolare dato.
\end{itemize}
Redux però ci pone davanti delle complicanze di carattere pratico.\newline
Nella sua versione base questo framework ci pone davanti ad una \emph{ingente quantità di boilerplate code}, difatti sono necessarie ingenti righe di codice con pochissime differenze per definire reducer e azioni.\newline
Inoltre risulta particolarmente ostica la creazione di middleware per la gestione di cambiamenti asincroni nello stato.\newline

\noindent Per ovviare a questo problema si è scelto di integrare l'utilizzo di \emph{Redux Toolkit} \cite{toolkit_redux}, che automatizza questo processo permettendoci di definire una struttura di più alto livello (\emph{Slice}) che viene poi convertita nelle strutture sopra citate.\newline
In particolare i middleware per i cambiamenti di stato asincroni, attraverso Redux Toolkit, vengono identificati dai \emph{Thunk} \cite{thunk_redux}.\newline
Utilizzando questa soluzione si è ottenuto un approccio per le azioni asincrone sicuro e si è evitato di incrementare ingentemente la quantità di codice del progetto, mantenendo una struttura ordinata e facilmente manutenibile.
