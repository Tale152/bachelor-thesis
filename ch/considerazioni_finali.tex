\chapter*{Considerazioni finali}
\addcontentsline{toc}{chapter}{Considerazioni finali} 
Data la natura molto ampia di questo progetto, saranno necessari ulteriori sforzi per ottenere una versione di NeuroFrame funzionante in ogni sua parte.\newline

\noindent D'altro canto, grazie ad un meticoloso processo di modellazione perpetuato dall'intera sezione IT di Vibre, i sottosistemi componenti il macrosistema NeuroFrame godono di un basso grado di dipendenza e ciò permette una valutazione ed uno sviluppo di ciascuno di essi.\newline

\noindent Dunque, ritenendo che il processo di ingegnerizzazione sia stato da noi svolto al meglio delle nostre capacità cercando di seguire le best practices della disciplina, vorrei trarre le conclusioni sul focus implementativo della mia esperienza: il sottosistema Dashboard.\newline

\noindent Riprendendo l'analisi dei requisiti del sistema, argomento della \emph{sezione 2.1}, tutti gli obbiettivi con priorità massima (Must have) concernenti la Dashboard sono stati raggiunti, ottenendo una prima versione funzionante dell'applicativo.\newline
Non sono comunque stati trascurati i requisiti con priorità minore da implementare in versioni future, creando un'architettura già predisposta a realizzare tali feature.\newline

\noindent Sviluppare il Simulatore, oltre a permettere l'effettivo accertamento dell'operatività della Dasbhoard, è stata un'esperienza molto formativa, permettendomi di mettere in atto soluzioni volte alla creazione di un tool general purpose che possa essere di valore per l'azienda anche in contesti che esulano dal mio progetto di tesi.\newline

\noindent Il prossimo step nel futuro di questo progetto vede la realizzazione dell'Applicativo di acquisizione secondo quanto modellato in questo elaborato, oltre che il perfezionamento definitivo del già ottimo algoritmo MindPulse.\newline

\noindent Credo molto nell'innovazione che questo progetto può apportare nel mondo del lavoro e spero che esso possa aiutare quanti più lavoratori possibili nelle sfide che affrontano ogni giorno.