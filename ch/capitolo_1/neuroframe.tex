\section{Neuroframe}
È normale, per un qualsiasi sportivo o lavoratore manuale, \emph{tenere sotto controllo il proprio livello di fatica fisica durante la giornata e prendere decisioni basandosi su di essa}; il lavoratore potrebbe decidere ad esempio se fare una pausa, quanta energia conservare per quel fondamentale scatto finale, se sollevare o no quel peso, se fare stretching...\newline
La sensoristica corporea che permette al nostro cervello di compiere queste scelte è numerosa, precisa e generalmente efficiente.\newline

\noindent Lo stesso discorso purtroppo non vale per tutti i cosiddetti Knowledge-workers (\emph{sezione 1.1}): i programmatori, i ricercatori, gli architetti, tutti coloro il cui lavoro richiede uno sforzo principalmente mentale piuttosto che fisico.\newline
Il cervello dovrebbe dunque monitorare il proprio funzionamento, ma non può farlo con efficacia: \emph{gli stessi sensori che dovrebbero controllare l'affaticamento vengono affaticati dallo sforzo!}\newline

\noindent Per questo motivo, i Knowledge-workers usano spesso strumenti esterni per controllare i propri sforzi, primi tra tutti orologi e tecniche di time-boxing (es. Pomodoro Timer).\newline
Il continuo crescere di casi di Occupational Burnout (\emph{sezione 1.1}) suggerisce che questi strumenti non bastano; è necessario fornire al cervello un equivalente, anche approssimativo, di quella sensoristica che esso è in grado di usare per le attività fisiche: una metrica per il carico cognitivo.\newline

\noindent Per risolvere questo problema è nato \emph{NeuroFrame}.\newline
Attraverso i modelli per il carico cognitivo e per l'affaticamento mentale di MindPulse (\emph{sezione 1.3}), NeuroFrame si propone come una \emph{suite Software as a Service} capace di monitorare lo stato di un numero arbitrario di persone (team) in real-time, con {\bf dashboard} capace di mostrare le metriche di ogni componente del team ed {\bf applicativi di acquisizione di segnali EEG} per ciascun individuo.\newline

\noindent Sulla base delle metriche mostrate tramite dashboard un \emph{Team manager} potrà prendere decisioni mirate ad incrementare il benessere cognitivo del team.\newline
NeuroFrame offrirà capacità di:
\begin{itemize}
  \item \emph{Prevenire i burnout}
  \item \emph{Valutare il carico cognitivo}
  \item \emph{Misurare l'affaticamento mentale}
\end{itemize}
\vspace{10mm}
\noindent Vibre si prepone dunque di portare questo servizio nelle aziende attraverso un prodotto scalabile e semplice da usare per l'utente finale, così da migliorare la produttività ed il benessere dei lavoratori nelle aziende.\newline
L'impresa non è tra le più semplici in quanto, come affrontato nella \emph{sezione 1.1}, il campo delle pBCI offre grandi potenzialità ma manca di un vasto parco servizi rivolto ad ambienti che esulano dallo sperimentale o dal medico.
