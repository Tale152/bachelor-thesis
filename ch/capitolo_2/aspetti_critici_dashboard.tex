\section{Aspetti critici dell'applicativo Dashboard}
Lo natura di questo applicativo pone davanti a delle problematiche e a conseguenti decisioni durante lo sviluppo.\newline
Alcune problematiche risultano \emph{legate intrinsecamente alla tecnologia implementativa}, cioè lo sviluppo di una soluzione che fa uso di tecnologie web (come specificato nei requisiti Should have \emph{sezione 2.1.3}) piuttosto che un applicativo nativo.\newline
Altre problematiche d'altro canto derivano da scelte \emph{legate alla user experience}.\newline
Segue una descrizione dei principali aspetti critici affrontati.
\subsection{Risorse computazionali limitate dei browser}
Scegliendo un approccio web piuttosto che un'applicazione nativa otteniamo un impoverimento dell'efficienza di calcolo, aggiungendo un layer extra tra l'applicativo e le risorse della macchina che lo eseguirà: il browser.\newline
Le prestazioni dell'applicazione dunque non saranno influenzate solamente dalla capacità di calcolo della macchina, ma anche dall'ottimizzazione del browser, le risorse che il sistema operativo concederà ad esso e da fattori non predicibili (come limitazioni imposte dal browser stesso o un alto numero di schede aperte nel browser).\newline
\emph{Al costo di una maggiore scalabilità e una compatibilità con un vasto parco dispositivi dunque si ottiene una riduzione delle performance che vincola ad uno sviluppo il più possibile ottimizzato}.
\subsection{Gestione di grandi quantità di dati}
\emph{La quantità di dati che descrivono l'evolversi dello stato cognitivo di una persona nel tempo presentano un volume e una densità molto elevati}, in quanto l'hardware che legge gli impulsi neurali (il caschetto Muse) effettua campionamenti del segnale con un periodo molto basso.\newline
Questa problema diventa ancora più preponderante all'aumentare del numero di componenti di un team.\newline
Risulta quindi necessario agire su più fronti:
\begin{itemize}
  \item
  {\emph{Ottimizzare al massimo gli algoritmi di processing dei dati nella dashboard}, così da processare velocemente i dati prima di richiedere al database un nuovo aggiornamento}
  \item
  {\emph{Ottimizzare le query di richiesta dati lanciate dalla dashboard verso il database}, al fine di ridurre il più possibile il tempo impiegato dal database ad eseguire la ricerca e per minimizzare la quantità di dati inviati nella rete}
  \item
  {\emph{Limitare il numero massimo di componenti di un team}}
  \item
  {\emph{Limitare se necessario la frequenza con cui l'app di acquisizione invia i dati alla piattaforma per l'elaborazione} scegliendo un buon compromesso tra accuratezza dei dati e performance}
\end{itemize}
\subsection{Rendering dei grafici}
Questa problematica risulta strettamente correlata alla precedente.\newline 
Mentre in precedenza il problema verteva maggiormente sui calcoli effettuati in locale sui dati ricevuti, qui si vuole sottolineare come \emph{il rendering dei grafici risulti molto oneroso computazionalmente}.\newline
Risulta quindi fondamentale un'\emph{ottimizzazione nella struttura e negli algoritmi che coinvolgono i dati utili al rendering}.
\subsection{Design vertically constrained}
La user interface che si occupa di mostrare i dati all'utente (la schermata dashboard) risulta vincolata da un requisito importante per l'efficacia pratica del sistema e per l'esperienza utente: \emph{al fine di permettere al team manager di poter interpretare velocemente la condizione del team è necessario che la maggior parte dei dati utili sia sempre visibile a schermo}.\newline
Questo ci porta ad avere delle costrizioni verticali (in altri termini tutto il contenuto verticale della pagina deve essere contenuto a schermo).\newline
Le tecnologie in ambito web utilizzano come punto di riferimento l'asse orizzontale della pagina, forti del fatto di poter estendere la pagina verticalmente potenzialmente all'infinito per poter rappresentare il contenuto.\newline
Risultano dunque necessarie delle soluzioni ad hoc per soddisfare il requisito preposto.\newline
In particolare, si è optato per tenere sempre visibili a schermo la barra delle opzioni, l'area dei singoli utenti e l'area del team nel complesso. Essendo il numero di componenti del team imprevedibile a priori, si è scelto di far scrollare verticalmente il contenuto dell'area degli utenti.
