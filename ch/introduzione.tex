\chapter*{Introduzione}
\addcontentsline{toc}{chapter}{Introduzione}
Nel corso degli anni sono stati compiuti passi da gigante nel campo delle neuroscienze, portando nel mondo reale idee che trovavano realizzazione soltanto nei romanzi di fantascienza.\newline

\noindent Seppur la strada sia ancora lunga, oggi siamo sempre più vicini ad avere una comprensione almeno parziale dei fenomeni che governano la nostra mente.\newline
Tali conoscenze, se sfruttate nel modo corretto, potrebbero portare a migliorare la condizione di vita di molte persone. \newline

\noindent Ma non sono soltanto le neuroscienze ad aver portato a termine grandi progressi; come ormai accade in gran parte delle discipline, l'informatica gioca un ruolo di supporto sempre più preponderante, fornendo nuove tecnologie ad uso e servizio dei campi scientifici ed umanistici più disparati.\newline
È quando neuroscienze ed informatica collaborano che ciò che sembrava impossibile trova un'opportunità per divenire reale.\newline

\noindent Non è raro che assottigliare il confine tra l'uomo e le macchine possa provocare un'iniziale paura nelle persone; il timore è che tali tecnologie ci possano allontanare dalla nostra natura umana portandoci a divenire esseri di natura artificiale.\newline
Seppur queste domande sorgano spontanee, ciò a cui bisogna guardare in realtà è l'obbiettivo con la quale una nuova tecnologia viene concepita.\newline

\noindent Ciò che si sta cercando di portare a termine non è la castrazione della nostra umanità, bensì l'ampliamento di essa aiutandola dove spesso i nostri limiti umani ci portano a provare disagi e sofferenze.\newline
Dopotutto anche un paio di occhiali da vista sono per definizione artificiali, ma non ci rendono meno umani.\newline
Attraverso di essi anzi possiamo godere a pieno della nostra vista sebbene il corso naturale degli eventi ci avrebbe relegato ad una vita dai toni sfocati.\newline

\vspace{10mm}
\noindent L'obbiettivo ultimo di questa tesi è la progettazione e lo sviluppo di un applicativo in grado di aiutare i lavoratori a soffrire il meno possibile la stanchezza mentale, migliorando il loro umore e, nel lungo andare, la loro salute.\newline
Un effetto secondario al benessere della psiche di un individuo sul lavoro, è l'incremento della produttività in quest'ultimo.\newline

\noindent Si viene dunque a creare una situazione in cui vincono tutti: da un lato i lavoratori sono più felici e rilassati e, dall'altro, l'azienda ne guadagna in flusso produttivo.\newline
Non ci sono dunque scuse, è impossibile evitare di considerare il valore che una tale soluzione porterebbe nelle vite dei lavoratori e delle aziende.\newline

\noindent In questa tesi dunque seguiremo lo sviluppo di tale soluzione dal punto di vista dell'ingegneria del software.\newline
Ci muoveremo dall'alto verso il basso; una volta compreso a pieno il contesto, il sistema verrà modellato e progettato nella sua interezza così da poterlo implementare scegliendo le soluzioni tecnologiche più adeguate.
